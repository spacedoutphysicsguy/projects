\documentclass[12pt,a4paper]{article}
\usepackage{amsmath, amssymb, graphicx, float, hyperref, geometry}
\geometry{margin=1in}
\title{S1 Coursework Comparison of Statistical Methods: Multi-Dimensional Likelihood Fitting vs. sWeights}
\author{ps2012}


\begin{document}

\maketitle
\begin{abstract}
This report compares the statistical power of two approaches for analyzing multi-dimensional probability distributions: a multi-dimensional likelihood fit and a weighted fit utilizing sWeights. Through mathematical derivations, numerical simulations, and parametric bootstrapping, we explore their advantages, drawbacks, and applications. All code and analysis are documented and reproducible.
\end{abstract}

\tableofcontents

\section{Introduction}
The goal of this study is to assess the performance of two statistical methods: the multi-dimensional likelihood fit and the weighted fit exploiting sWeights. This involves mathematical proofs, visualization of distributions, parameter estimation, and a thorough comparison of the two approaches under various conditions. The Crystal Ball distribution serves as a central component of the analysis, modeling the signal probability density function (PDF).

\section{Mathematical Proof of Normalization Constant \texorpdfstring{$N$}{N}}
\label{sec:proof}
We prove that the normalization constant $N^{-1}$ for the Crystal Ball PDF can be written as:
\begin{equation}
N^{-1} = \sigma \left[ \frac{m}{\beta(m-1)} e^{-\beta^2/2} + \sqrt{2\pi} \Phi(\beta) \right].
\end{equation}
The derivation includes step-by-step justification, utilizing the properties of Gaussian and power-law tails.

\section{Definition and Normalization of PDFs}
\label{sec:pdfs}
This section describes the signal and background models:
\begin{itemize}
    \item Signal PDF in $X$: Crystal Ball distribution, truncated in $[0,5]$.
    \item Signal PDF in $Y$: Exponential decay with parameter $\lambda$.
    \item Background PDFs: Uniform in $X$ and truncated Gaussian in $Y$.
\end{itemize}
Normalization of these PDFs is verified numerically through integration.

\section{Visualization of Distributions}
\label{sec:visualization}
Figures in this section include:
\begin{itemize}
    \item 1D projections of signal, background, and total PDFs in $X$ and $Y$.
    \item A 2D plot of the joint probability density function.
\end{itemize}
These visualizations confirm the correctness of the models.

\section{Sampling and Maximum Likelihood Fitting}
\label{sec:fitting}
\subsection{Sampling from the Joint PDF}
A high-statistics sample of $100,000$ events is generated using the defined PDFs. Numerical methods and the `timeit` library are used to measure execution time for:
\begin{itemize}
    \item Sampling $100,000$ events.
    \item Performing extended maximum likelihood fits.
\end{itemize}
\subsection{Parameter Estimation}
The extended maximum likelihood fit estimates the nine parameters of the model, including uncertainties.

\section{Parametric Bootstrapping}
\label{sec:bootstrapping}
Using parametric bootstrapping, we:
\begin{itemize}
    \item Generate ensembles of samples of varying sizes ($500$, $1000$, $2500$, $5000$, $10000$).
    \item Assess bias and uncertainty in the decay constant $\lambda$ as a function of sample size.
\end{itemize}
The results highlight trends in statistical precision and potential biases.

\section{sWeights Analysis}
\label{sec:sweights}
The sWeights method is applied by fitting only the $X$ variable and projecting the signal density in $Y$. Weighted samples are used to estimate $\lambda$. The bias and uncertainty are compared with those from parametric bootstrapping.

\section{Comparison of Methods}
\label{sec:comparison}
\subsection{Advantages and Drawbacks}
\begin{itemize}
    \item \textbf{Multi-dimensional likelihood fit}: High statistical power but computationally intensive.
    \item \textbf{sWeights}: Efficient for partial models but prone to biases if weights are not accurately estimated.
\end{itemize}
\subsection{Preferred Scenarios}
Recommendations are made for when each method is appropriate, based on the complexity of the problem and computational constraints.

\section{Conclusion}
\label{sec:conclusion}
This study demonstrates the trade-offs between multi-dimensional likelihood fitting and sWeights. While the former offers higher precision, the latter can be advantageous in high-dimensional problems with limited computational resources. Future work could explore hybrid approaches to leverage the strengths of both methods.

\section*{References}
\begin{itemize}
    \item Lecture notes and references cited in the coursework.
    \item Relevant documentation for libraries used (e.g., scipy, iminuit).
\end{itemize}

\end{document}
